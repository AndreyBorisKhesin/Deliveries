\input{header}
\physics
\begin{document}
\papertitle{Optimizing the Distribution of Goods}
\paperauth{C}{Cheng}{University of Toronto}
\paperauth{A}{Khesin}{University of Toronto}
\paperdate{April 30th, 2018}
\begin{paperabs}
Abstract goes here
\end{paperabs}
\begin{paper}
\papersec{Introduction}

The presented problem asks what is the best way to optimize the distribution of
goods, given certain constraints, without knowing where they will be needed and
when.
For this task, we designed and tested several algorithms, by implementing a
program to check how the algorithms perform, given a random set of customers.
The customers were in turn generated by the program from the provided
spreadsheet of historical data with the corresponding random noise.

In solving this problem several important assumptions were made, which are
listed below.
\begin{enumerate}
\item
Trucks, when starting their route from the warehouse, are only allowed to
deliver items to a shop, pick up items at a plant, or both in that order, after
which they must return to the warehouse they departed from.
\item
The number of customers for a given shop, for a given product is determined
as follows.
First the total number of customers that visited that shop looking for that
product in a given month was calculated.
This value was modified by a randomly chosen coefficient between 0.9 and 1.1.
Then, the resulting value was rounded and that number of customers was randomly
distributed among the business days of that month.
\item
Products cost nothing to produce, seeing as the price was
not mentioned anywhere in the problem statement.
This could be interpreted as the fact that the factory workers have fixed
contracts, so that their wages will be expenses regardless of the amount
produced.
Additionally, there is no limit to the amount of products that can be produced
in a day.
\item
After the sales for a given day had ended, an assesment of the current stock
could be made, and additional trucks ordered for that same night at the extra
rate.
\item
If a customer walks into a store and sees that the item they want is not
available, they are told to come by the next day, and there is a 20\%
probability that they do not come back, and an 80\% probability that they do.
\end{enumerate}

In the above list, the second rule has the random 10\% fluctuation applied to
the total number of customers in a month, as the daily number of customers is
typically so small that any value within 10\% will just round to the average,
erasing any variance in the number of customers from day to day.

\papersec{Method}

The key to our approach was noticing the large disparity in cost between the
prices of a heater and an air conditioner, compared to the expenses of driving
delivery trucks, even when considered with a 20\% penalty.
In fact, the price of a single heater is enough to cover even the longest of
delivery routes to a single shop.
This means that even missing out on a single sale is more costly than paying the
extra money to make sure that a shop is fully stocked.

\papersec{Data}

Data goes here

\papersec{Analysis}

Analysis goes here

\papersec{Conclusion}

Conclusion goes here

\end{paper}
\end{document}
